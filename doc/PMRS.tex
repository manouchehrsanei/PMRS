\documentclass[a4paper]{article}

\usepackage[english]{babel}
\usepackage[utf8]{inputenc}
\usepackage{amsmath}
\usepackage{graphicx}
\usepackage[colorinlistoftodos]{todonotes}
\usepackage[backend=biber,style=authoryear,autocite=inline]{biblatex}
\bibliography{references/references.bib} 

%%%%%%%%%%%%%%%%%%%%%%%%%%%%%%%%%%%%%%%%%%%%%%%%%%%%%%%%%%%%%%%%%%%%%%%%%%%%%%%
% Commands
\newcommand{\PMRS}{\textit{PMRS }}
\newcommand{\neopz}{\textit{neopz }}

\title{Permeability Multiplier Reservoir Simulator \PMRS project}

\author{Sanei M. and Duran O. and Devloo P.}

\date{\today}

\begin{document}
\maketitle

\begin{abstract}
This document contains details about the implementation and formulations for a novel upscaling approach to consider permeability variations in conventional reservoir simulations.
\end{abstract}

\section{Introduction}
\label{sec:introduction}

\section{Fundamentals}
\subsection{sec:fundamentals}

\section{Upscaling process for permeability considering geomechanic effects}
\subsection{sec:upscaling}

\section{Programming and neopz implementation}
\label{sec:programmaing}

\PMRS uses plastic materials already implemented inside \neopz library.  
\cite{SouzaNeto2008}.

\subsection{About the custom memory materials}


The word impl here is an abbreviation for implementation

\section{Verifications}

\subsection{2D simulations}

\subsection{3D simulations}

\section{Results}

\section{References}
\printbibliography


\end{document}